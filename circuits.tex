\documentclass[12pt]{article}
 
\usepackage{amssymb,amsfonts,amsmath}
\usepackage{bm}
\usepackage{dsfont}
\usepackage{color,soul}
\usepackage{graphicx}
\usepackage{caption}

\usepackage{float}
\usepackage{hyperref}

\usepackage{tikz}
\usetikzlibrary{backgrounds,fit,decorations.pathreplacing,calc}

\newcommand{\ket}[1]{\ensuremath{\left| #1 \right \rangle}}
\newcommand{\bra}[1]{\ensuremath{\left \langle #1 \right |}}
\newcommand{\braket}[2]{\ensuremath{\left\langle #1\left|#2 \right.\right\rangle}}
\def\ketbra#1#2{{\vert#1\rangle\!\langle#2\vert}}

\newcommand{\1}[1]{\mathds{1}\left[#1\right]}
\DeclareMathOperator{\Tr}{Tr}

\newtheorem{remark}{Remark}
\newtheorem{definition}{Definition}
\newtheorem{theorem}{Theorem}
\newtheorem{lemma}{Lemma}
\newtheorem{example}{Example}

\usepackage{bbm}

\newcommand{\eye}{\mathds{1}} % vector space identity op
\usepackage[dvipsnames]{xcolor}
\usepackage[colorinlistoftodos,textsize=small,backgroundcolor=BlueGreen,linecolor=Blue]{todonotes}


\title{Factoring $k$-controlled-unitaries into $4k^2$ controlled gates without ancillas}
\date{\today}

\author{Jacob Biamonte$^1$, Nike Dattani$^2$ and Mauro E.S.~Morales$^1$\\$^1$Deep Quantum Labs\\ Skolkovo Institute of Science and Technology\\ 3 Nobel Street, 
Moscow, Russia 121205\\
$^2$HPQC Labs \\ National Research Council of Canada \\ 100 Sussex Drive, Ottawa, Canada K1A 0R6}

%\homepage{DeepQuantum.AI}

\begin{document}
\maketitle

\subsection{The naive approach}

\begin{itemize}
    \item $12(k-1)$ 2-qubit gates, 
    \item $18(k-1)$ 1-qubit gates, 
    \item $k-1$ auxiliary qubits
    
\end{itemize}

\begin{figure}
  \centerline{
    \begin{tikzpicture}[thick]
    %
    % `operator' will only be used by Hadamard (H) gates here.
    % `phase' is used for controlled phase gates (dots).
    % `surround' is used for the background box.
    \tikzstyle{operator} = [draw,fill=white,minimum size=1.5em] 
    \tikzstyle{phase} = [fill,shape=circle,minimum size=5pt,inner sep=0pt]
    \tikzstyle{surround} = [fill=blue!10,thick,draw=black,rounded corners=2mm]
    %
    % Qubits
    \node at (0,0) (q1) {\ket{c_1}};
    \node at (0,-1) (q2) {\ket{c_2}};
    \node at (0,-2) (q3) {\ket{c_3}};
    %
    % Column 1
    \node[phase] (phase1) at (1,0) {} ;
    \node[operator] (op21) at (1,-1) {H} edge [-] (q2);
    \node[operator] (op31) at (1,-2) {H} edge [-] (q3);
    %
    % Column 3
    \node[phase] (phase11) at (2,0) {} edge [-] (op11);
    \node[phase] (phase12) at (2,-1) {} edge [-] (op21);
    \draw[-] (phase11) -- (phase12);
    %
    % Column 4
    \node[phase] (phase21) at (3,0) {} edge [-] (phase11);
    \node[phase] (phase23) at (3,-2) {} edge [-] (op31);
    \draw[-] (phase21) -- (phase23);
    %
    % Column 5
    \node[operator] (op24) at (4,-1) {H} edge [-] (phase12);
    \node[operator] (op34) at (4,-2) {H} edge [-] (phase23);
    %
    % Column 6
    \node (end1) at (5,0) {} edge [-] (phase21);
    \node (end2) at (5,-1) {} edge [-] (op24);
    \node (end3) at (5,-2) {} edge [-] (op34);
    %
    % Bracket
    \draw[decorate,decoration={brace},thick] (5,0.2) to
	node[midway,right] (bracket) {$\frac{\ket{000}+\ket{111}}{\sqrt{2}}$}
	(5,-2.2);
    %
    % Background Box
    \begin{pgfonlayer}{background} 
    \node[surround] (background) [fit = (q1) (op31) (bracket)] {};
    \end{pgfonlayer}
    %
    \end{tikzpicture}
  }
  \caption{
    Nike will fill this in.
  }
\end{figure}


\bibliography{refs}
\bibliographystyle{plain}
\end{document}